\selectlanguage{german}
\chapter{Anforderungen}\label{chap:Anforderungen}
\chapterstart
In diesem Kapitel stelle ich die Anforderungen an das System vor. Der Fokus liegt bei den für das Thema relevanten Anforderungen, es wird keine vollständige Anforderungsanalyse durchgeführt.
\section{Allgemein}
Wie schon Eingangs erwähnt, soll das System Bilder in regelmäßigen Abständen von Kameras abholen und diese auf einem Archivserver speichern. Die Abholung erfolgt dabei - je nach Standort der Kamera - einmal alle Stunden oder einmal jede Minute.\\
Da die Kameras getauscht werden und neue Kameras hinzukommen ist es notwendig, dass die Einrichtung der einzelnen Aufgaben für den Benutzer einfach und auch während des Betriebes erfolgen kann.\\
Die im folgenden angeführten Anforderungen sind vom System auf alle Fälle zu erfüllen. Die Beschreibung der Anforderungen folgt dem Schema von User-Stories.
\section{Funktionelle Anforderungen}
Im folgenden sind die wichtigsten funktionellen Anforderungen an den Scheduler angeführt:
\begin{enumerate}
	\item Das System kann Aufgaben ausführen die aus einer Aktion (Aufgabentyp) und einer zu der Aktion gehörenden Konfiguration gehören.
	\item Das System kann beliebig viele Aufgaben verwalten und diese bestmöglich zu den geforderten Zeitpunkten ausführen.
	Die Anzahl der Aufgaben sollte zumindest 100.000 sein.
	\item Als Benutzer kann ich für jede Aufgabe ein oder mehrere Zeitpläne festlegen in dem die Aufgabe abgearbeitet wird. Der Zeitplan muss folgende Einteilungen zulassen
	\begin{enumerate}
		\item Einmal zu einem festgelegtem Zeitpunkt
		\item Periodisch alle x Sekunden
		\item Stündlich
		\item Täglich zu einer festgelegten Uhrzeit
		\item Wöchentlich an einem festgelegtem Tag und einer festgelegten Uhrzeit
	\end{enumerate}
	\item Das System muss verschiedene Aufgaben unterstützen. Der Scheduler muss ohne Änderungen in der Lage sein verschiedene Aufgabentypen zu erledigen. 
	\item Das System muss nicht echtzeitfähig sein, eine Abweichung der tatsächlichen Ausführungszeitpunkten gegenüber den vorgegebenen Zeitpunkten ist erlaubt. Die Aufgaben sollten jedoch mit einer möglichst geringen Abweichung zu den definierten Zeitpunkten abgearbeitet werden.
	\item Das System stellt die Ergebnisse der Ausführung dem Benutzer als einsehbares Log zur Verfügung.
	 \item Das System kann Aufgaben auf mehreren Rechnern ausführen. Die Verteilung erfolgt dabei automatisch durch den Scheduler. Neue Rechner können dem Scheduler ohne Änderung der Konfiguration hinzugefügt werden.
	 \item Ein Ausfall eines Rechners wird durch das System automatisch erkannt und behoben. Die Ausführung von Aufgaben wird dann automatisch auf alle andern Rechner verteilt. Die auf dem Rechner zum Zeitpunkt der Deaktivierung bearbeiteten Aufgaben sollen bestmöglich wiederhergestellt und erneut ausgeführt werden. Aufgaben die nicht wiederhergestellt werden können .
	 \item Einzelne Rechner sollen nur bestimmte Aufgabentypen ausführen. Dadurch können für die jeweiligen Typen spezialisierte Rechner verwendet werden.
\end{enumerate}
\section{Technische Anforderungen}
Im folgenden sind die wichtigsten technischen Anforderungen an den Scheduler angeführt:
\begin{enumerate}
	\item Der Scheduler muss unter Microsoft Windows Server lauffähig sein.
	\item Die Aufgaben sind als Klassen definiert.
	\item Die auf einem Rechner zur Verfügung stehenden Klassen müssen ohne Anpassung des Scheduler Codes oder der Konfiguration änderbar sein.
	\item Der Scheduler kann mehrere Aufgaben pro Rechner gleichzeitig verarbeiten. Eine Synchronisation gegenüber den Ressourcen die von den Aufgaben benötigt werden ist nicht in der Verantwortung des Schedulers.
	\item Der Administrator kann die Anzahl der Prozessoren die pro Rechner für den Scheduler verwendet werden konfigurieren.
\end{enumerate}

\chapterend
