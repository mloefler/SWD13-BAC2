\selectlanguage{german}
\chapter{Einleitung}\label{chap:Einleitung}
\chapterstart
In diesem Kapitel stelle ich das zugrundeliegende Problem einführend vor. Aus einer allgemeinen Beschreibung werden im Zusammenspiel mit einer beispielhaften Verwendung der zu entwickelnden Software die grundlegenden Anforderungen erarbeitet. Aus diesen werden die Ziele dieser Arbeit definiert.
\section{Motivation}
Die meisten Benutzer von Computerprogrammen nutzen diese interaktiv. Sie befinden sich in einem Dialog mit der Software wo sie aufgrund ihrer Eingabe eine bestimmte Ausgabe erwarten. Handelt es sich bei den Aufgaben die so dem Programm gestellt werden um länger laufende Aktionen, so gibt der Benutzer bloß den Anstoß, die Aufgabe selbst wird ohne weitere Interaktion erledigt und das Ergebnis angezeigt. Die meiste Zeit ist die Aufmerksamkeit des Benutzers nicht notwendig. Gibt es nun mehre Aufgabenstellungen dieser Art die auch noch wiederholt ausgeführt werden müssen, so sollten diese automatisiert werden, so dass sie ohne Eingriff des Benutzers ausgeführt werden und dieser nur mehr bei Problemen informiert wird.\\
Bei Aufgaben dieser Art handelt es sich oft auch um Datenimporte, bei denen aus unterschiedlichen Quellsystemen Daten geladen und in Zielsysteme importiert oder mit den Daten in Zielsystemen abgeglichen werden.\\
Die regelmäßige Ausführung von Programmen wird durch \emph{Scheduler} übernommen. Dies sind Programme die gemäß einem durch den Benutzer definierten Zeitplan Aufgaben automatisiert ausführen. Zwei der bekanntesten Vertreter sind \emph{cron} unter Linux oder der \emph{Windows Task Scheduler} für Microsoft Windows.\\
Beide Scheduler bieten eine gute Möglichkeit Programme basierend auf Zeitplänen auszuführen, beide haben jedoch die Einschränkungen dass sie
\begin{itemize}
	\item Programme nur auf einem Rechner ausführen und
	\item komplette Programme ausführen.
\end{itemize}
Wenn die Anzahl der Aufgaben die ausgeführt werden soll ansteigt, müssen die Aufgaben durch den Benutzer selbst auf mehrere Rechner verteilt werden. Wenn ein Rechner ausfällt oder ein neuer Rechner hinzukommt muss der Benutzer die Aufgaben ebenso neu verteilen.\\
Ebenso sind beide Scheduler darauf ausgelegt komplette Programme, also \emph{executables} auszuführen. Dadurch müssen diese für Ausführung ein neuer \emph{Prozess} angelegt werden und die dazugehörigen Daten erneut in den Speicher geladen werden. Werden nun häufig die selben Aufgaben erledigt und sind diese relativ klein, so steht der Zusatzaufwand des Ausführens meist nicht im Verhältnis zur Ausführungszeit der Aufgabe selbst.\\
Für den Fall der umzusetzen war stellten diese beiden Herausforderung ein großes Hindernis dar. Nach der Anforderungsanalyse und einer Machbarkeitsstudie wurde daher entschieden einen eigenen Scheduler umzusetzen.

\section{Beispielhafte Verwendung}
Eines der Hauptthemen in der Industrie 4.0 ist die Einbindung von Sensoren in EDV Systeme um durch deren Daten einerseits Informationen zu aktuellen (Betriebs-) Zuständen zu erhalten und andererseits durch die Sammlung der Daten mithilfe von Big-Data Algorithmen weitere Schlussfolgerungen ziehen zu können.\parencite[S. 36ff]{Manzei2015}\\
Dabei wird in der Regel stillschweigen von der neuesten Generation von Sensoren ausgegangen, die ihrerseits aktiv Daten in die Cloud publizieren.\footnote{Siehe zum Beispiel \parencite{ms_azureiot} "...the device sends..."} Diese Sensoren sind aktiv und lösen somit die weitere Verarbeitung der Daten aus. Es ist keine Aktivierung von Außen notwendig.\\
Viele der derzeit bestehenden Sensoren sind allerdings passiv und stellen ihre Daten nur nach Nachfrage zur Verfügung. Die Daten dieser Sensoren müssen also periodisch abgefragt und an die weiterverarbeitenden Prozesse weitergereicht werden.\\
In der beispielhaften Verwendung geht es darum Standbilder von bereits bestehenden Überwachungskameras zu Dokumentationszwecken in einem Archiv abzulegen. Die Anzahl der Kameras beträgt in etwa 1.000 und wird mit der Zeit deutlich steigen, damit muss das System frei skalieren.\\ 
\section{Ziele der Arbeit}
Ziel der Arbeit ist es einen funktionstüchtigen Scheduler der auf mehreren Rechnern läuft und nicht vollständige Prozesse ausführt zu erstellen. Im Rahmen dieser Arbeit wird nur der Scheduler Kern entwickelt, die Oberflächen zum Einrichten und Verwalten der Aufgaben, sowie das Einsehen der Protokolle wird nicht umgesetzt.
\\Im Theorieteil werden die zugrundeliegenden Themen im Bereich Datenbanken und Concurrency\footnote{Zu Deutsch Nebenläufigkeit. Im Sinne der besseren technischen Verständlichkeit werde ich im folgenden nur den englischen Fachbegriff verwenden.} mittels Literaturrecherche erläutert.\\ Durch diese Erläuterungen sollen die getroffenen Designentscheidungen untermauert werden und einem Entwickler die Themenkomplexe verständlich vermittelt werden.

\section{Verwendete Programmierumgebung}
Für die Umsetzung des Schedulers - und in folge auch der Implementierung der einzelnen Aufgaben - wurde das Microsoft .NET Ökosystem gewählt. Verwendet wurden
\begin{itemize}
	\item Microsoft .NET 4.6
	\item Microsoft C\# 7
	\item Microsoft Visual Studio 2017 Express Edition
	\item Microsoft Entity Framework 6 als ORM Schicht\footnote{object-relational mapping, eine Bibliothek die den Zugriff auf eine relationale Datenbank sowie die Abbildung der Daten in Klassen und Instanzen implementiert. Vgl. \parencite{ef_orm} }
	\item Microsoft SQL Server 2016 als Datenbank
\end{itemize}
Die Auswahl wurde einerseits durch meinen bisherige Erfahrung mit .NET / C\# geprägt, anderseits spielten vor allem folgende Faktoren eine entscheidende Rolle:
\begin{itemize}
	\item Lauffähigkeit unter Microsoft Windows
	\item Automatische Installierbarkeit von Sicherheitspatches für die Laufzeitumgebung
	\item Hohe Ausführungsgeschwindigkeit unter Microsoft Windows
	\item Mögliche Portierbarkeit auf andere Betriebssysteme mittels Microsoft .Net Core das zu diesem Zeitpunkt bereits in einer ersten Version zur Verfügung steht.
\end{itemize}
\section{Anforderungen}\label{chap:Anforderungen}
\subsection{Allgemein}
Wie schon Eingangs erwähnt, soll das System Bilder in regelmäßigen Abständen von Kameras abholen und diese auf einem Archivserver speichern. Die Abholung erfolgt dabei - je nach Standort der Kamera - einmal alle Stunden oder einmal jede Minute.\\
Da die Kameras getauscht werden und neue Kameras hinzukommen ist es notwendig, dass die Einrichtung der einzelnen Aufgaben für den Benutzer einfach und auch während des Betriebes erfolgen kann.\\
Die im folgenden angeführten Anforderungen sind vom System auf alle Fälle zu erfüllen. Die Anforderungen sind hierarchisch dargestellt und auf Spezifikationsebene 1 und 2 beschränkt\parencite[S. 45]{rupp2009}.
\subsection{Funktionelle Anforderungen}
Im folgenden sind die wichtigsten funktionellen Anforderungen an den Scheduler angeführt:
\begin{enumerate}
	\item Der Benutzer kann Aufgaben im Scheduler anlegen
	\begin{enumerate}
		\item Der Benutzer kann eine Aufgabe eines bestimmten Typs mit einem bestimmten Kontext (Konfiguration) anlegen.
		\item Der Scheduler kann beliebig viele Aufgaben verwalten und diese bestmöglich zum geforderten Zeitpunkt erledigen. Die Anzahl der Aufgaben muss dabei zumindest größer als 10.000 sein.
		\item Als Benutzer kann ich für jede Aufgabe ein oder mehrere Zeitpläne festlegen in dem die Aufgabe abgearbeitet wird. Der Zeitplan muss folgende Einteilungen zulassen:
		\begin{enumerate}
			\item Einmal zu einem festgelegtem Zeitpunkt
			\item Periodisch alle x Sekunden
			\item Stündlich
			\item Täglich zu einer festgelegten Uhrzeit
			\item Wöchentlich an einem festgelegtem Tag und einer festgelegten Uhrzeit
		\end{enumerate}
	\end{enumerate}
	\item Der Scheduler kann die Aufgaben auf mehrere Rechner verteilt ausführen.
	\begin{enumerate}
		\item Der Administrator kann durch Installation der Software einfach weitere Rechner in den Scheduler einbinden. Er braucht dazu keinerlei globale Konfiguration ändern.
		\item Jeder Rechner kann unterschiedliche Typen von Aufgaben übernehmen. Dadurch ist eine Spezialisierung der Rechner Hardware für einzelne Aufgabentypen möglich.
		\item Die Umsetzung von weiteren Typen an Aufgaben benötigt keine Änderung am Scheduler. Diese neuen Implementierungen brauchen nur auf den Rechnern installiert werden. Ein Neustart des Schedulers zur Aktivierung ist akzeptabel.
		\item Die Verteilung von Aufgaben erfolgt automatisch durch den Scheduler.
		\item Bei einem Ausfall werden die Aufgaben von den verbleibenden Rechnern bestmöglich ausgeführt.
		\item Die auf dem Rechner zum Zeitpunkt der Deaktivierung bearbeiteten Aufgaben sollen bestmöglich wiederhergestellt und erneut ausgeführt werden.
	\end{enumerate}
	\item Der Scheduler führt die Aufgaben möglichst nahe am geplanten Zeitpunkt aus.
	\begin{enumerate}
		\item Der Scheduler muss nicht echtzeitfähig sein, eine Abweichung der tatsächlichen Ausführungszeitpunkten gegenüber den vorgegebenen Zeitpunkten ist erlaubt. Die Aufgaben sollten jedoch mit einer möglichst geringen Abweichung zu den definierten Zeitpunkten abgearbeitet werden.
	\end{enumerate}
	\item Der Benutzer erhält Information über den aktuellen Status des Schedulers
	\begin{enumerate}
		\item Der Benutzer sieht welche Aufgaben im System definiert sind.
		\item Der Benutzer sieht wann welche Aufgaben zuletzt ausgeführt wurden und wann sie wieder geplant sind.
		\item Der Benutzer sieht den Status der einzelnen Ausführungen der Aufgaben und kann Fehler erkennen.
	\end{enumerate}
\end{enumerate}
\subsection{Technische Anforderungen}
Im folgenden sind die wichtigsten technischen Anforderungen an den Scheduler angeführt:
\begin{enumerate}
	\item Der Scheduler muss unter Microsoft Windows Server lauffähig sein.
	\item Die Aufgaben sind als Klassen definiert.
	\item Die auf einem Rechner zur Verfügung stehenden Klassen müssen ohne Anpassung des Scheduler Codes oder der Konfiguration änderbar sein.
	\item Der Scheduler kann mehrere Aufgaben pro Rechner gleichzeitig verarbeiten. Eine Synchronisation gegenüber den Ressourcen die von den Aufgaben benötigt werden ist nicht in der Verantwortung des Schedulers.
	\item Der Administrator kann die Anzahl der Prozessoren die pro Rechner für den Scheduler verwendet werden konfigurieren.
\end{enumerate}
\chapterend
