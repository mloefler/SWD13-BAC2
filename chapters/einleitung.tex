\selectlanguage{german}
\chapter{Einleitung}\label{chap:Einleitung}
\chapterstart
In diesem Kapitel stelle ich das zugrundeliegende Problem einführend vor. Aus einer allgemeinen Beschreibung werden im Zusammenspiel mit einer beispielhaften Verwendung der zu entwickelnden Software die grundlegenden Anforderungen erarbeitet. Aus diesen werden die Ziele dieser Arbeit definiert.
\section{Motivation}
Die meisten Benutzer von Computerprogrammen nutzen diese interaktiv. Sie befinden sich in einem Dialog mit der Software wo sie aufgrund ihrer Eingabe eine bestimmte Ausgabe erwarten. Handelt es sich bei den Aufgaben die so dem Programm gestellt werden um länger laufende Aktionen, so gibt der Benutzer bloß den Anstoß, die Aufgabe selbst wird ohne weitere Interaktion erledigt und das Ergebnis angezeigt. Die meiste Zeit ist die Aufmerksamkeit des Benutzers nicht notwendig. Gibt es nun mehre Aufgabenstellungen dieser Art die auch noch wiederholt ausgeführt werden müssen, so sollten diese automatisiert werden, so dass sie ohne Eingriff des Benutzers ausgeführt werden und dieser nur mehr bei Problemen informiert wird.\\
Bei Aufgaben dieser Art handelt es sich oft auch um Datenimporte, bei denen aus unterschiedlichen Quellsystemen Daten geladen und in Zielsysteme importiert oder mit den Daten in Zielsystemen abgeglichen werden.\\
Die regelmäßige Ausführung von Programmen wird durch \emph{Scheduler} übernommen. Dies sind Programme die gemäß einem durch den Benutzer definierten Zeitplan Aufgaben automatisiert ausführen. Zwei der bekanntesten Vertreter sind \emph{cron} unter Linux oder der \emph{Windows Task Scheduler} für Microsoft Windows.\\
Beide Scheduler bieten eine gute Möglichkeit Programme basierend auf Zeitplänen auszuführen, beide haben jedoch die Einschränkungen dass sie
\begin{itemize}
	\item Programme nur auf einem Rechner ausführen und
	\item komplette Programme ausführen.
\end{itemize}
Wenn die Anzahl der Aufgaben die ausgeführt werden soll ansteigt, müssen die Aufgaben durch den Benutzer selbst auf mehrere Rechner verteilt werden. Wenn ein Rechner ausfällt oder ein neuer Rechner hinzukommt muss der Benutzer die Aufgaben ebenso neu verteilen.\\
Ebenso sind beide Scheduler darauf ausgelegt komplette Programme, also \emph{executables} auszuführen. Dadurch müssen diese für Ausführung ein neuer \emph{Prozess} angelegt werden und die dazugehörigen Daten erneut in den Speicher geladen werden. Werden nun häufig die selben Aufgaben erledigt und sind diese relativ klein, so steht der Zusatzaufwand des Ausführens meist nicht im Verhältnis zur Ausführungszeit der Aufgabe selbst.\\
Für den Fall der umzusetzen war stellten diese beiden Herausforderung ein großes Hindernis dar. Nach der Anforderungsanalyse und einer Machbarkeitsstudie wurde daher entschieden einen eigenen Scheduler umzusetzen.

\section{Beispielhafte Verwendung}
Eines der Hauptthemen in der Industrie 4.0 ist die Einbindung von Sensoren in EDV Systeme um durch deren Daten einerseits Informationen zu aktuellen (Betriebs-) Zuständen zu erhalten und andererseits durch die Sammlung der Daten mithilfe von Big-Data Algorithmen weitere Schlussfolgerungen ziehen zu können.\cite[S. 36ff]{Manzei2015}\\
Dabei wird in der Regel stillschweigen von der neuesten Generation von Sensoren ausgegangen, die ihrerseits aktiv Daten in die Cloud publizieren.\footnote{Siehe zum Beispiel \cite{ms_azureiot} "...the device sends..."} Diese Sensoren sind aktiv und lösen somit die weitere Verarbeitung der Daten aus. Es ist keine Aktivierung von Außen notwendig.\\
Viele der derzeit bestehenden Sensoren sind allerdings passiv und stellen ihre Daten nur nach Nachfrage zur Verfügung. Die Daten dieser Sensoren müssen also periodisch abgefragt und an die weiterverarbeitenden Prozesse weitergereicht werden.\\
In der beispielhaften Verwendung geht es darum Standbilder von bereits bestehenden Überwachungskameras zu Dokumentationszwecken in einem Archiv abzulegen. Die Anzahl der Kameras beträgt in etwa 1.000 und wird mit der Zeit deutlich steigen, damit muss das System frei skalieren.\\ 
\section{Ziele der Arbeit}
Ziel der Arbeit ist es einen funktionstüchtigen Scheduler der auf mehreren Rechnern läuft und nicht vollständige Prozesse ausführt zu erstellen. Im Rahmen dieser Arbeit wird nur der Scheduler Kern entwickelt, die Oberflächen zum Einrichten und Verwalten der Aufgaben, sowie das Einsehen der Protokolle wird nicht umgesetzt.
\\Im Theorieteil werden die zugrundeliegenden Themen im Bereich Datenbanken und Concurrency\footnote{Zu Deutsch Nebenläufigkeit. Im Sinne der besseren technischen Verständlichkeit werde ich im folgenden nur den englischen Fachbegriff verwenden.} mittels Literaturrecherche erläutert.\\ Durch diese Erläuterungen sollen die getroffenen Designentscheidungen untermauert werden und einem Entwickler die Themenkomplexe verständlich vermittelt werden.

\section{Verwendete Programmierumgebung}
Da eine der technischen Anforderungen die Ausführbarkeit unter Microsoft Windows ist, habe ich mich zur Umsetzung der Lösung mittels C\# und dem Microsoft .Net Framework v 4.6 entschlossen, als Datenbank wird Microsoft SQL Sever 2016 verwendet. Die Entwicklung erfolgte unter Microsoft Visual Studio 2017. Als objektrationales Mapping Tool zur einfachen Anbindung von Datenbanken wurde das Microsoft Entity Framework 6.1 verwendet.\\
Die Entscheidung war durch folgende Askepte geprägt
\begin{itemize}
	\item Lauffähigkeit unter Microsoft Windows
	\item Automatische Installierbarkeit von Sicherheitspatches für die Laufzeitumgebung
	\item Hohe Ausführungsgeschwindigkeit unter Microsoft Windows
	\item Mögliche Portierbarkeit auf andere Betriebssysteme mittels Microsoft .Net Core das zu diesem Zeitpunkt bereits in einer ersten Version zur Verfügung steht.
\end{itemize}
\chapterend
