\selectlanguage{german}
\begin{abstract}
	
Die wiederkehrende Erledigung von Aufgaben ist in der EDV ein häufig anzutreffendes Problem. Dabei müssen die Aufgaben in etwa zu einem bestimmten Zeitpunkt in der Zukunft erledigt werden oder wiederkehrend in vorgegebenen Zeitabständen ausgeführt werden - ohne Interaktion eines Benutzers.\\
Im Internet der Dinge werden solche Systeme auch oft eingesetzt um Daten von Geräten zu holen die diese nicht aktiv zur Verfügung stellen können. Im Internet der Dinge kann diese Anzahl dabei rasch hoch werden, das System muss mit der wachsenden Anzahl an Geräten in der Leistungsfähigkeit mitwachsen und die erwarteten Tätigkeiten ausführen.\\
In dieser Arbeit wird ein solches System das sowohl funktionell als auch in der Leistung skaliert entwickelt und prototypisch eingesetzt. Einige der wichtigen Themen werden auch in der Theorie erklärt und die Umsetzung im Code erläutert. Schwerpunkt dieser Erläuterungen sind insbesondere die Problemstellungen bei der Persistenz der durchzuführenden Aufgaben und der Sicherstellung der Leistung.\\
Die Umsetzung wird zur Lösung einer tatsächlichen Aufgabenstellung eingesetzt. Der Theorieteil basiert auf einer Literaturrecherche.\\
Zielpublikum der Arbeit sind Entwickler die an Serversystemen arbeiten die wiederkehrende Aufgeben zu erledigen haben oder ähnliche Problemstellungen im Bereich der Skalierung zu erfüllen haben.
\end{abstract}


\selectlanguage{english}
\begin{abstract}

Recurring tasks are a common challenge in software systems. Tasks either need to be scheduled to be executed at a future point in time or need to be executed in set intervals without further user interaction.\\
The Internet of things often uses such systems to fetch data from devices that can not actively push the data. The number of devices within the Internet of things is often large and heterogeneous, so the tasks that need to be run are diverse. The system therefore needs to be able to scale in performance and functionality.\\
As a part of this thesis such a system is developed put to use in a real environment. Exemplary problems like persistence and concurrency are elaborated in this thesis. The theoretical part is backed by literature research.
The target audience of this theses are developers that either face the challenge of recurring tasks or are generally interested in the scaling problems.

\end{abstract}
\selectlanguage{german}