\selectlanguage{german}
\chapter{Anforderungen}\label{chap:Anforderungen}
\chapterstart
In diesem Kapitel stelle ich die Anforderungen an das System vor. Der Fokus liegt bei den für das Thema relevanten Anforderungen, es wird keine vollständige Anforderungsanalyse durchgeführt.
\section{Allgemein}
Wie schon Eingangs erwähnt, soll das System Bilder in regelmäßigen Abständen von Kameras abholen und diese auf einem Archivserver speichern. Die Abholung erfolgt dabei - je nach Standort der Kamera - einmal alle Stunden oder einmal jede Minute.\\
Da die Kameras getauscht werden und neue Kameras hinzukommen ist es notwendig, dass die Einrichtung der einzelnen Aufgaben für den Benutzer einfach und auch während des Betriebes erfolgen kann.\\
Die im folgenden angeführten Anforderungen sind vom System auf alle Fälle zu erfüllen. Die Anforderungen sind hierarchisch dargestellt und auf Spezifikationsebene 1 und 2 beschränkt\cite[S. 45]{rupp2009}.
\section{Funktionelle Anforderungen}
Im folgenden sind die wichtigsten funktionellen Anforderungen an den Scheduler angeführt:
\begin{enumerate}
	\item Der Benutzer kann Aufgaben im Scheduler anlegen
	\begin{enumerate}
		\item Der Benutzer kann eine Aufgabe eines bestimmten Typs mit einem bestimmten Kontext (Konfiguration) anlegen.
		\item Der Scheduler kann beliebig viele Aufgaben verwalten und diese bestmöglich zum geforderten Zeitpunkt erledigen. Die Anzahl der Aufgaben muss dabei zumindest größer als 10.000 sein.
		\item Als Benutzer kann ich für jede Aufgabe ein oder mehrere Zeitpläne festlegen in dem die Aufgabe abgearbeitet wird. Der Zeitplan muss folgende Einteilungen zulassen:
		\begin{enumerate}
			\item Einmal zu einem festgelegtem Zeitpunkt
			\item Periodisch alle x Sekunden
			\item Stündlich
			\item Täglich zu einer festgelegten Uhrzeit
			\item Wöchentlich an einem festgelegtem Tag und einer festgelegten Uhrzeit
		\end{enumerate}
	\end{enumerate}
	\item Der Scheduler kann die Aufgaben auf mehrere Rechner verteilt ausführen.
	\begin{enumerate}
		\item Der Administrator kann durch Installation der Software einfach weitere Rechner in den Scheduler einbinden. Er braucht dazu keinerlei globale Konfiguration ändern.
		\item Jeder Rechner kann unterschiedliche Typen von Aufgaben übernehmen. Dadurch ist eine Spezialisierung der Rechner Hardware für einzelne Aufgabentypen möglich.
		\item Die Verteilung von Aufgaben erfolgt automatisch durch den Scheduler.
		\item Bei einem Ausfall werden die Aufgaben von den verbleibenden Rechnern bestmöglich ausgeführt.
		\item Die auf dem Rechner zum Zeitpunkt der Deaktivierung bearbeiteten Aufgaben sollen bestmöglich wiederhergestellt und erneut ausgeführt werden.
	\end{enumerate}
	\item Der Scheduler führt die Aufgaben möglichst nahe am geplanten Zeitpunkt aus.
	\begin{enumerate}
		\item Der Scheduler muss nicht echtzeitfähig sein, eine Abweichung der tatsächlichen Ausführungszeitpunkten gegenüber den vorgegebenen Zeitpunkten ist erlaubt. Die Aufgaben sollten jedoch mit einer möglichst geringen Abweichung zu den definierten Zeitpunkten abgearbeitet werden.
	\end{enumerate}
	\item Der Benutzer erhält Information über den aktuellen Status des Schedulers
	\begin{enumerate}
		\item Der Benutzer sieht welche Aufgaben im System definiert sind.
		\item Der Benutzer sieht wann welche Aufgaben zuletzt ausgeführt wurden und wann sie wieder geplant sind.
		\item Der Benutzer sieht den Status der einzelnen Ausführungen der Aufgaben und kann Fehler erkennen.
	\end{enumerate}
\end{enumerate}
\section{Technische Anforderungen}
Im folgenden sind die wichtigsten technischen Anforderungen an den Scheduler angeführt:
\begin{enumerate}
	\item Der Scheduler muss unter Microsoft Windows Server lauffähig sein.
	\item Die Aufgaben sind als Klassen definiert.
	\item Die auf einem Rechner zur Verfügung stehenden Klassen müssen ohne Anpassung des Scheduler Codes oder der Konfiguration änderbar sein.
	\item Der Scheduler kann mehrere Aufgaben pro Rechner gleichzeitig verarbeiten. Eine Synchronisation gegenüber den Ressourcen die von den Aufgaben benötigt werden ist nicht in der Verantwortung des Schedulers.
	\item Der Administrator kann die Anzahl der Prozessoren die pro Rechner für den Scheduler verwendet werden konfigurieren.
\end{enumerate}

\chapterend
